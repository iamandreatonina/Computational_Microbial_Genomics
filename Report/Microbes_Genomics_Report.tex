\documentclass[a4paper,titlepage, oneside]{book}
\usepackage[signatures, swapnames,nouppercase]{frontespizio}
\usepackage[english]{babel}
\usepackage[a4paper, top=2cm, bottom=2cm, left=1.5cm, right=1.5cm]{geometry}
\usepackage{graphicx}
\usepackage{amsmath} % or simply amstext
\usepackage{subfig}
\usepackage{graphicx}
\usepackage{hyperref}
\usepackage{xcolor}
\definecolor{light-gray}{gray}{0.95}
\newcommand{\code}[1]{\colorbox{light-gray}{\texttt{#1}}}
\makeatletter
\def\@makechapterhead#1{%
  \vspace*{50\p@}%
  {\parindent \z@ \raggedright
    \normalfont
    \interlinepenalty\@M
    \Huge \bfseries\thechapter\space  #1\par\nobreak
    \vskip 40\p@
  }}
\makeatother
\newcommand{\angstrom}{\text{\normalfont\AA}}
\begin{document}
\begin{frontespizio}
\Istituzione{}
\Logo[1.5cm]{C:/Users/acer/Desktop/Tirocinio/Teoria/imm\_tn.png}
\Divisione{BACHELOR’S DEGREE IN}
\Scuola{Biomolecular Sciences and Technology}
\Titolo{Molecular Dynamics analysis to study the interaction between the human cytomegalovirus glycoprotein US9 and MICA*008}
\NCandidato{Graduant}
\Candidato{Gloria Lugoboni}
\NRelatore{Supervisor}{}
\Relatore{Prof. Emiliano Biasini}
\NCorrelatore{Tutor}{}
\Correlatore{Dr. Marta Rigoli}
\Punteggiatura{}
\Piede{Date of discussion 21/07/2022}
\end{frontespizio}

\tableofcontents

\chapter{Introduction}

\section{Motivation}
The main goal of this report is to present the general workflow that can be used when performing a metagenome analysis. Specifically, here we present the investigation of the uSGB15132 that was identified thanks to a previous study of the human microbiome \cite{SGB2}.
This report contains the main findigs on the uSGB15132 that were obtained thanks to a process of genome annotation, followed by pangenome and phylogenetic analysis, succeeded by taxonomic characterization to identify the specific strain encoded in the genomes of the MAGs.



\section{Metagenome Sequencing, Assembly, and Binning}
Metagenome sequencing enables the construction of metagenomes-assembled genomes (MAGs). A MAG can be seen as a microbial genome obtained by a preliminary passage of genome assembly of high quality contigs. This kind of analysis enables us to identify novel species thanks to a passage of annotation and taxonomic classification \cite{MAG}.

A typical shotgun metagenomic process involves a specific pipeline, a step of sample acquisation and sequencing, a step of assembly or mapping and finally a step of binning followed by genome-annotation. This whole process is than completed with further, more complete analysis \cite{Metagenome}.



The selection of high quality bins enables the identification of MAGs, characterized by a high completness and low levels of contamination, therefore, they can be used to operate taxonomic annotation and gene prediction \cite{MAG2}. These can be grouped together in the same species genome bin (SGB) if they exceed a certain threshold of nucleotide identity, fixed at 95\% \cite{SGB2, SGB}.



With the term pangenome, we indicate the union of the \textit{core genome}, containing genes present in all strains, and the \textit{dispensable genome}, also called \textit{accessory genome}, containing genes present in two or more strains and genes unique to single strains \cite{Medini}.

Eventually, thanks to phylogenetic analysis and taxonomic characterization it is possible to contestualize microbial genomes and determine their genetic and phenotype relationships \cite{Phylo}.




\chapter{Methods}
\section{Softwares and parameters used}
\subsection{Genome annotation (Prokka)}

Prokka is a fast and accurate command line software tool used to annotate prokaryoyic genomes.  It produces standardized output files that can be used for further analysis or viewing using genome browsers.

Prokka expect one single input file in a FASTA format, containing an assembled genome. The process of annotation is possible thanks to the comparison of the gene codes with a large database of known sequences, identifying the best match as the most significative one and therefore associating the labelling and the relevant features to the gene codes. Prokka use this method in an hierarchical manner, using initially small and reliable databases moving only at the end of the process to protein family databases.
Prokka produces several outputs file, listed in the Figure \ref{fig:prokka-output}  \cite{Prokka}.

We need to specify several parameters, specifically, the input files, our MAGs; the output directory \code{--outdir} and the parameter \code{--kingdom Bacteria} that is needed to specify the annotation mode, to make the annotation more fast.
\\ \newline \\
\code{prokka --kingdom Bacteria --outdir SGB15132\_prokka\_output .f*}.

%such as the .fna file that represent a complete fasta file and contains the genome of the organism; the .faa file, that contains the amminoacids of the encoded proteins; the .tsv file that is a table summary of the annotations of the proteins with a description; finally, a .gbk file, a genome bank file.

\begin{figure}[ht]
\centering
\includegraphics[scale=0.6]{Table\_prokka.png}
\caption{Prokka outputs files \cite{Prokka}.}
\label{fig:prokka-output}
\end{figure}

\subsection{Pangenome analysis and Phylogenetic analysis (Roary, Roary + FastTree)}

Roary is a tool that enables the construction of large-scale prokaryotic pangenomes, identifying the core and accessory genes.

The input file to Roary is a GFF file containing sequences features. 

Roary collects the coding regions from the annotated input genome. It operates a clustering process creating a network and defining a phylogenetic tree. A matrix is therefore obtained and the pangenome (core genes and accessory genes) is defined.  The process of clustering is based on the minimum percentage of identity, setted to 95\% by default \cite{Roary}.

The main output of Roary is a tree obtained using the presence and absence of the accessory genes. It is a tree used to have an initial insight of the data, grouping in a quick way the genomes based on their accessory genes \cite{Roary}. The tree can be visualized using iTOL, an online tool for phylogenetic tree display \cite{iTOL}.

Roary returns three graphs, the newick tree associated to the pangenome table (a matrix of core and accessory genes), a pie chart of the breakdown of genes and the number of isolate genomes are present in, a graph with the frequency of genes versus the number of genomes. \cite{Roary-outputs}


There are some main parameters that need to be specified to roary, specifically, the input \code{.gff} files; the output directory \code{-f roary\_out}; the \code{-i} parameter, specifing the percentage identity used by blastp, here used at 95\%; the \code{-cd} parameter, percentage of isolates that a gene must be in to be considered part of the core genome, here setted at 90\%.
\\ \newline \\ \code{roary .gff -i 95 -cd 90} \\ \newline \\


With Roary it is also possible to perform a core gene alignment to generate a more reliable tree. A core-genome alignment is more scalable, respect to a whole-genome alignment, is useful to identify the core genes conserved in all genomes and can be used to infer the phylogeny \cite{Core-align}.

The main parameters to be specified are the \code{-e} parameter, needed to perform a core gene alignment; the \code{-n} parameter, to use MAFFT as the tool for the multiple sequence alignmet, making the process faster; and finally, the parameter \code{-p}, needed to specify the number of threads, increasing therefore the speed \cite{Roary}.
\\
\newline
\\
\code{roary .gff -i 95 -cd 90 -e -n -p 8}.
\\
\newline
\\
The core gene aligment can be used to construct a phylogenetic tree. This is possible using FastTree, a tool for constructing large phylogenies, estimating their reliability. FastTree exploit Neighbor-Joining and nearest neighbor interchanges to create a phylogentic tree \cite{FastTree}.
Specifically, we used FastTreeMP, that allow the parallelization of the steps needed in computing a tree \cite{FTMP}.
The tree is obtained using the following code,\\ \newline \\ \code{FastTreeMP -gtr -nt -out core\_gene.tre core\_gene\_alignment.aln}\\ \newline \\ in which the parameter \code{-gtr} express the generalized time-reversible model (to be used with nucleotide alignments only) while the parameter \code{-nt} is used to specify that the alignment is performed on nucleotides.




\subsection{Taxonomic assignment (PhyloPhlAn 3.0)}

PhyloPhlAn 3.0 is an accurate and rapid tool to perform microbial genome characterization and phylogenetic analysis of metagenomes. PhyloPhlAn 3.0 can integrate public genome information to the genomes in input with a taxonomic resolution at the species and strain level. It allow the assignation of the closest species-level genome to each of the bin obtained via metagenomic assembly \cite{Phylo}.
There are some main parameters needed to be specified, mainly, the input folder with the \code{-i} parameter; the output folder, with the parameter \code{-o}; the \code{--nproc} parameter, used to specify the CPUs that can be used; the \code{-n} parameter that allow us to decide  how many SGBs (sorted by increasing average genomic distance) will be reported for each input bin in the output file; the \code{--database\_update} parameter to update the databases file, the \code{-d} parameter to specify the name of the output database and finally, the \code{--verbose} parameter to print to the bash \cite{PhyloGuide}.
The final command is the following:\\ \newline \\ \code{phylophlan\_metagenomic -i phylo -o phylo\_out --nproc 4 -n 1 --database\_update -d CMG2324 --verbose}

\chapter{Results and discussion}

\section{uSGB 15132}

We were provided with a set of 31 high-quality prebinned metagenomes grouped in the same uSGB labelled SGB15132.

As shown in the Figure \ref{fig:compl} the bins have a completeness higher that 97.3 and the maximum redundacy registered is equal to 2.25.
The MAGs were associated to a study condition, as shown in the Figure \ref{fig:studyCondition}, we used this information to interpret the analyisis that were effectuated later on.

\begin{figure}[ht]
\centering
\includegraphics[scale=0.6]{completeness\_distribution.png}
\caption{Completeness distribution of the given MAGs.}
\label{fig:compl}
\end{figure}


\begin{figure}[ht]
\centering
\includegraphics[scale=0.6]{Study\_condition.png}
\caption{Pie chart of the study condition of our MAGs}
\label{fig:studyCondition}
\end{figure}

\section{Genome annotation}
After the genome annotation process, possible using prokka, we were able to identify that, the number of the CDS (protein coding sequence) is slightly variable, spanning from a minimun value of 2651 to a maximum of 3935. For each MAG, as shown in the Figure \ref{fig:CDS} more or less a half of the CDS are known proteins, while the other half is represented by RNA or hypotetical proteins.


\begin{figure}[ht]
\centering
\includegraphics[scale=0.4]{CDS\_hypothetical\_knowns.png}
\caption{Main annotation results}
\label{fig:CDS}
\end{figure}

\section{Pangenome analysis}



Each SGB strain was found to contain an average of 1317 genes that are present in every strain (core genome), plus 8407 genes that are absent in more than one strains (strain $<$ 31) (accessory genome). This can be seen in the Figure \ref{fig:pangenome2}, showing the frequency plot of the genes per genome, this plot gives a general overview of the frequency of genes within a whole genome set, typically these plots have an U shape and most genes can be detected in a single genome or in all genomes \cite{Analysis-roary}. Here it is shown that the number of genes present in all the genomes correspond to the number of core genes.
The accessory genes are also divided into genes present in less than 4 strains (cloud genome) or genes present in 4 or more strains but not all strains (shell genome) \cite{Medini}. The figure \ref{fig:pangenome5} show the subparts constituting the pangenome, associated with the total number of genes while the Figure \ref{fig:pangenome1} show the tree obtained via the analysis associated to a heatmap in which we can easily define the core genes that are present in all the genomes. It is important to remember that the tree is not obtained with a complete clustering methodology.


\begin{figure}[ht]
\centering
\includegraphics[scale=0.6]{Graphs/pangenome\_frequency.png}
\caption{Pangenome Frequency}
\label{fig:pangenome2}
\end{figure}

\begin{figure}[ht]
\centering
\includegraphics[scale=0.3]{pangenome\_matrix.png}
\caption{Pangenome matrix}
\label{fig:pangenome1}
\end{figure}

\begin{figure}[ht]
\centering
\includegraphics[scale=0.5]{Graphs/pangenome\_pie.png}
\caption{Core and accessory genome}
\label{fig:pangenome5}
\end{figure}


Looking at the Figures \ref{fig:pangenome3} and  \ref{fig:pangenome4} we can derive an idea on the pangenome, specifically if it is open or closed. It is important to remember that these results were obtained using only 31 metagenomes, making the process of discussion more complex. Looking at the Conserved vs Total genes plot in Figure \ref{fig:pangenome3} the total genes initial slope is very sharp and only when almost all the genomes are added, the slope start to decrease. The conserved gene line is almost stationary, with little changes when the genomes are all added. Taking in consideration only this graph will be leading us to think that the pangenome is open, therefore, the observation of the Unique vs New genes plot can be useful to support this hypotesis or not. Looking at the Figure \ref{fig:pangenome4}, we can see that, adding new genomes doesn't give new information, indeed the number of new genes is almost at zero when we add the last genome. This help us to assume that the pangenome is possibly closed but we need more genomes to have a better characterization.


\begin{figure}[ht]
\centering
\includegraphics[scale=0.4]{Graphs/conserved\_vs\_total\_genes.png}
\caption{Conserved vs Total genes}
\label{fig:pangenome3}
\end{figure}

\begin{figure}[ht]
\centering
\includegraphics[scale=0.4]{Graphs/unique\_vs\_new\_genes.png}
\caption{Unique vs New genes}
\label{fig:pangenome4}
\end{figure}

\section{Phylogenetic analysis }
The phylogenetic analysis enabled us to obtain phylogenetic trees both form the presence and absence of the accessory genes and the alignment of the core genes.
The Figure \ref{fig:Tree1} represent the tree obtained using the accessory genome data, while the Figure \ref{fig:Tree2} represent the tree obtained using the core genome data.\\Each branch is associated to a confidence level from 0 to 1, obtained by the metadata. We also divided the branches based on the country and the diseases as shown in the legends.
\\Operating a comparison between the two we can see that in both trees we have an enanched division based on the countries (westernized and non westernized), with a stronger clustering in the Tree \ref{fig:Tree1}, since the tree is obtained using the data from the accessory genomes it is easier to observe a "clustering-specific" based on the westernization countries, diet and culture differences.
\\Regarding the disease information, we think the the Tree \ref{fig:Tree2} is more explenatory. We can specifically see that there is a branching point after which the non-healty samples are grouped almost all togheter. We hypotize that this difference could be related to DNA mutation that were acquired after the activation of pathogenic pathways.

\begin{figure}[ht]
\centering
\includegraphics[scale=0.4]{tree\_plots/alignment\_tree.png}
\caption{Tree from accessory genome information}
\label{fig:Tree1}
\end{figure}


\begin{figure}[ht]
\centering
\includegraphics[scale=0.5]{tree\_plots/alignment\_tree\_ft.png}
\caption{Tree from core genome alignment.}
\label{fig:Tree2}
\end{figure}




\section{Taxonomic association}
PhyloPhlAn enable us to operate a phylogenetic analysis. The main output correspond to a list of the closest SGBs sorted by their increasing average sequence distance (Mash distance) in a tab-separated file \cite{PhyloGuide}. 
%The information of each SGB contains specific columns:
%\\
%\\ \code{my\_bin}: is the input bin name
%\\ \code{(k|u)SGB\_ID}: the SGB ID, \textit{k} indicate a known SGB, \textit{u} indicate an unknown SGB
%\\ \code{taxa\_level}: taxonomic level the SGB has been assigned to
%\\ \code{taxonomy}: the full taxonomic label assigned to the SGB
%\\ \code{average\_mash\_distance}: the distance is calculated with respect to all the genomes in the SGB.
%\\
\\In our case, for each MAG we get the same taxa\_level, the species Flavonifractor plautii, a strictly anaerobic rod shaped bacterium. It is a commensal of the human intestinal microbiota which is  usually difficult to isolate from samples \cite{Plauti}.
\\Thanks to further invastigation on this bacteria it was found that this strain was recently obtained via the unification of two previuosly distinct species, specifically, Clostridium orbiscindens and Eubacterium plautii \cite{Unification}.
\\It is known from literature that this species is correleated with hip joint infection \cite{HipJoint}, chronic kidney disease, various autoimmune disorders and epithelial invasive potential \cite{PlautiiDiseases}. Finally, this strain was associated with blood infections in immunosuppressed patients \cite{PlautiiBlood},
we think that this findings are to keep in consideration for further analysis on this strain, since the 33\% of our samples were obtained from patients with correlated diseases like rheumatoid arthritis (a kind hip joint infection) and hypertension.



\chapter{Conclusion}
Thanks to this project it was possible to obtain an insight on the metagenome workflow and the main analysis that are neeed when dealing with unkown genomes.
We initially investigated the uSGB we were given computing general statistics on the quality and the associated metadata.
We then performed a process of gene annotation and retrived informations about the number of coding sequences and hypotetical proteins in these CDS (Figure \ref{fig:CDS}).
We than were able to perform a pangenome analysis and hypotize that the pangenome of our uSGB is closed as the addition of new genomes gives us little to no information as shown in the Figure \ref{fig:pangenome4}. Also, it was possible to identify the genes shared by at least 90\% of our MAGs, the core genome. These correspons to $\sim$ 15\% of the total gene.
After the tree obtained with a phylogenetic analysis (\ref{fig:Tree1}, \ref{fig:Tree2}) we hypotized that the obtained tree could be useful to investigate the custering based on the country of origin of the sample and the presence of absence of specific diseases.
From the taxonomix assigment of the uSGB it was possible to identify the \textit{Oscillospiraceae} family and specifically the \textit{Flavonifractor plautii} species, correlated with variuos diseases and infections \cite{HipJoint,PlautiiDiseases, PlautiiBlood} that could correlate with the ones observed in our samples. More specific analysis should be effectuated to better analyze the clustering of the non-healty and healty samples and the role of westernization and diet, keeping in mind that there could be a bias derived from the fact that all the control samples were obtained from westernized patients while most of the diseased samples were obtained from not westernized patients.
\vfill
Link to the github page with the project: \textit{https://github.com/iamandreatonina/Microbiomal\_Genomics\_Segata}




\begin{thebibliography}{}

\bibitem{SGB2}
Pasolli E, Asnicar F, Manara S, Zolfo M, Karcher N, Armanini F, Beghini F, Manghi P, Tett A, Ghensi P, Collado MC, Rice BL, DuLong C, Morgan XC, Golden CD, Quince C, Huttenhower C, Segata N. \emph{Extensive Unexplored Human Microbiome Diversity Revealed by Over 150,000 Genomes from Metagenomes Spanning Age, Geography, and Lifestyle.} Cell. 2019 Jan 24;176(3):649-662.e20. \\doi: 10.1016/j.cell.2019.01.001.

\bibitem{MAG}
Yang, Chao et al. \emph{A review of computational tools for generating metagenome-assembled genomes from metagenomic sequencing data,} Computational and structural biotechnology journal vol. 19 6301-6314. 23 Nov. 2021, \\doi:10.1016/j.csbj.2021.11.028

\bibitem{Metagenome}
Thomas T, Gilbert J, Meyer F.
\emph{Metagenomics - a guide from sampling to data analysis.} Microb Inform Exp. 2012 Feb 9;2(1):3. \\doi: 10.1186/2042-5783-2-3. PMID: 22587947; PMCID: PMC3351745.






\bibitem{MAG2}
Chao Yang, Debajyoti Chowdhury, Zhenmiao Zhang, William K. Cheung, Aiping Lu, Zhaoxiang Bian, Lu Zhang,
\emph{A review of computational tools for generating metagenome-assembled genomes from metagenomic sequencing data,} Computational and Structural Biotechnology Journal, Volume 19, 2021, Pages 6301-6314, ISSN 2001-0370, \\https://doi.org/10.1016/j.csbj.2021.11.028.


\bibitem{SGB}
Blanco-Míguez, A., Beghini, F., Cumbo, F. et al. \emph{Extending and improving metagenomic taxonomic profiling with uncharacterized species using MetaPhlAn 4.} Nat Biotechnol (2023). \\https://doi.org/10.1038/s41587-023-01688-w


\bibitem{Medini}
Medini, Duccio et al. \emph{The microbial pan-genome.} Current opinion in genetics and development vol. 15,6 (2005): 589-94. \\doi:10.1016/j.gde.2005.09.006


\bibitem{Phylo}
Asnicar, F., Thomas, A.M., Beghini, F. et al.
\emph{Precise phylogenetic analysis of microbial isolates and genomes from metagenomes using PhyloPhlAn 3.0.} Nat Commun 11, 2500 (2020).\\ https://doi.org/10.1038/s41467-020-16366-7

\bibitem{Prokka}
Torsten Seemann,\emph{Prokka: rapid prokaryotic genome annotation,} Bioinformatics, Volume 30, Issue 14, July 2014, Pages 2068–2069,\\ https://doi.org/10.1093/bioinformatics/btu153

\bibitem{Roary}
Page AJ, Cummins CA, Hunt M, Wong VK, Reuter S, Holden MT, Fookes M, Falush D, Keane JA, Parkhill J. \emph{Roary: rapid large-scale prokaryote pan genome analysis.} Bioinformatics. 2015 Nov 15; 31(22):3691-3. \\doi: 10.1093/bioinformatics/btv421

\bibitem{iTOL}
Ivica Letunic, Peer Bork, \emph{Interactive Tree Of Life (iTOL) v5: an online tool for phylogenetic tree display and annotation}, Nucleic Acids Research, Volume 49, Issue W1, 2 July 2021, Pages W293–W296, \\https://doi.org/10.1093/nar/gkab301

\bibitem{Roary-outputs}
\emph{https://sanger-pathogens.github.io/Roary/}

\bibitem{Core-align}
Treangen, T.J., Ondov, B.D., Koren, S. et al. \emph{The Harvest suite for rapid core-genome alignment and visualization of thousands of intraspecific microbial genomes.} Genome Biol 15, 524 (2014).\\https://doi.org/10.1186/s13059-014-0524-x

\bibitem{FastTree}
Morgan N. Price, Paramvir S. Dehal, Adam P. Arkin, \emph{FastTree: Computing Large Minimum Evolution Trees with Profiles instead of a Distance Matrix, Molecular Biology and Evolution,} Volume 26, Issue 7, July 2009, Pages 1641–1650, \\https://doi.org/10.1093/molbev/msp077


\bibitem{FTMP}
Price MN, Dehal PS, Arkin AP, \emph{FastTree 2–approximately maximum-likelihood trees for large alignments}. PLoS One. 2010, 5\\doi: e9490-10.1371/journal.pone.0009490.


\bibitem{PhyloGuide}
\emph{https://github.com/biobakery/phylophlan/wiki}


\bibitem{Analysis-roary}
\emph{https://help.ezbiocloud.net/gene-frequency-plot-in-pan-genome/}

\bibitem{Plauti}
Berger FK, Schwab N, Glanemann M, Bohle RM, Gärtner B, Groesdonk HV. \emph{Flavonifractor (Eubacterium) plautii bloodstream infection following acute cholecystitis.} IDCases. 2018 Oct 28;14:e00461. \\doi: 10.1016/j.idcr.2018.e00461.

\bibitem{Unification}
Carlier, Jean-Philippe et al. \emph{Proposal to unify Clostridium orbiscindens Winter et al. 1991 and Eubacterium plautii (Séguin 1928) Hofstad and Aasjord 1982, with description of Flavonifractor plautii gen. nov., comb. nov., and reassignment of Bacteroides capillosus to Pseudoflavonifractor capillosus gen. nov., comb. nov.} International journal of systematic and evolutionary microbiology vol. 60,Pt 3 (2010): 585-590.\\doi:10.1099/ijs.0.016725-0

\bibitem{HipJoint}
Wilton, Alexander et al. \emph{Periprosthetic Hip Joint Infection with Flavonifractor plautii: A Literature Review and Case Report.} Hip and pelvis vol. 34,4 (2022): 255-261.\\doi:10.5371/hp.2022.34.4.255

\bibitem{PlautiiDiseases}
Liu RT, Rowan-Nash AD, Sheehan AE, Walsh RFL, Sanzari CM, Korry BJ, Belenky P. \emph{Reductions in anti-inflammatory gut bacteria are associated with depression in a sample of young adults.} Brain Behav Immun. 2020 Aug;88:308-324.\\doi: 10.1016/j.bbi.2020.03.026. Epub 2020 Mar 27. 

\bibitem{PlautiiBlood}
Costescu Strachinaru, Diana Isabela et al. \emph{A case of Flavonifractor plautii blood stream infection in a severe burn patient and a review of the literature.} Acta clinica Belgica vol. 77,3 (2022): 693-697.\\doi:10.1080/17843286.2021.1944584

\end{thebibliography}




\end{document}